\chapter{Exercise 1}

In the first exercise a class Point was implemented. The developed code can be found in the Appendix (\ref{lst:point}). The following paragraphs will reason about different aspects of the code and why things were solved the way they were.

The class Point represents a point by its cartesian coodinates. For this the variables x and y were choosen. As described in the exercise the class should have two constructors.

\begin{lstlisting}[
    language=java,
    caption=Constructor method headers for class Point]
    //default ctor with x=0.0, y=0.0
    public Point(){}

    //parametricized ctor
    public Point(double x, double y){}
\end{lstlisting}

Further methods for normalizing, rotating and displacing the point are given by the following methods.

\begin{lstlisting}[
    language=java,
    caption=Methods in class Point]
    //calculates distance from origin to point (normalizing vector)
    public double norm(){}

    //rotates point around origin by theta degrees
    public void rotate(double theta){}

    //moves the point by amount p.x and p.y
    public void displace(Point p){}
\end{lstlisting}

Also the methods hashcode, equals and toString were overriden to coorespond to the defined behaviour. In the following section some reason is given on the specific implementation for each of these methods.

\section{Method hashCode}
As a hashing algorithm a very basic and simple default is provided by eclipse. This can be described by the following equation.

\begin{displaymath}
    hash(p.x) = prime \cdot 1 + (x \oplus (x \gg 32))
\end{displaymath}
\begin{displaymath}
    hash(p) = prime \cdot hash(p.x) + (y \oplus (y \gg 32))
\end{displaymath}

Since only the values of x and y are used and no other random aspects can occur during this calculation the value for to points will be equal, if they are equal as defined in the equals method.

\section{Method equals}
The method equals was overridden to check the values of x and y. If those are the same equals() returns true, else it returns false. There are no checks for when equals is called with something other then another point, as this should be done by the caller before. The check \lstinline[language=java]{if(this == obj)} is done for faster comparison of the same object.
To safely compare the values of x and y, the following code is used.
\begin{lstlisting}[language=java, caption=Safely compare double values in java]
    if(Double.doubleToLongBits(this.x) != Double.doubleToLongBits(other.x)) {...}
\end{lstlisting}

This was crucial as in Java \lstinline[language=java, breaklines=true]{Double.NaN == Double.NaN} is false and nearly all error handling was done by setting the values of x and y to NaN.

\lstinline[language=java, breaklines=true]{Double.doubleToLongBits(Double.NaN) == Double.doubleToLongBits(Double.NaN)} on the other hand is true.

\section{Method toString}
As the output format of toString() was given bei the exercise, the default toString method was overridden. As the commata seperator was \textbf{.} instead of \textbf{,} (which is used in germany) the output locale was set in Code.
\begin{lstlisting}[language=java, caption=Setting locale to get correctly printed decimal seperator]
    String.format(Locale.ENGLISH, "...", x, y);
\end{lstlisting}
This overrides the systemwide set locale and corresponds to the correct commata seperator. Also the values of x and y are printed in scientific notation with four decimals.
